
\documentclass[11pt]{article}
\begin{document}
\section{pg 106: \#1}
    ID main \\
    OPAREN \\
    CPAREN \\
    OCURLY \\
    CONST \\
    TYPE float \\
    ID payment \\
    EQUALS \\
    FLOATVALUE 384.00 \\
    SEMICOLON \\
    TYPE float \\
    ID bal \\
    SEMICOLON \\
    TYPE int \\
    ID month \\
    EQUALS \\
    INTVALUE 0 \\
    SEMICOLON \\
    ID bal \\
    EQUALS \\
    INTVALUE \\
    SEMICOLON \\
    WHILE \\
    OPAREN \\
    ID bal \\
    GREATERTHAN \\
    INTVAL \\
    CPAREN \\
    OCURLY \\
    ID printf \\
    OPAREN \\
    STRVALUE "Month: \%2d Balance: \%10.2f\textbackslash n" \\
    COMMA \\
    ID month \\
    COMMA \\
    ID bal \\
    CPAREN \\
    SEMICOLON \\
    ID bal \\
    EQUALS \\
    ID bal \\
    MINUS \\
    ID payment \\
    PLUS \\
    FLOATVALUE 0.015 \\
    STAR \\
    ID bal \\
    SEMICOLON \\
    ID month \\
    EQUALS \\
    ID month \\
    PLUS \\
    INTVALUE l \\
    SEMICOLON \\
    CCURLY \\
    CCURLY \\
\section{pg 106 \#4}
    a. (a$\mid$(bc)*d)+ \\
    \begin{tabular}{|c|c|c|c|c|}
  \hline  & a & b & c & d \\ \hline
        1 & 4 & 2 &   & 4 \\ \hline
        2 &   &   & 3 &   \\ \hline
        3 &   & 2 &   & 4 \\ \hline
       *4 & 4 & 2 &   & 4 \\ \hline
    \end{tabular}

    b. ((0$\mid$1)*(2$\mid$3)+)$\mid$0011 \\
    \begin{tabular}{|c|c|c|c|c|}
   \hline & 0 & 1 & 2 & 3 \\ \hline
        1 & 2 & 6 & 7 & 7 \\ \hline
        2 & 3 & 6 & 7 & 7 \\ \hline
        3 & 6 & 4 & 7 & 7 \\ \hline
        4 & 7 & 5 & 7 & 7 \\ \hline
       *5 &   &   & 7 & 7 \\ \hline
        6 & 6 & 6 & 7 & 7 \\ \hline
       *7 &   &   & 7 & 7 \\ \hline
    \end{tabular}

    c. (aNot(a))*aaa \\
        Note: Not(a) depends on what $\Sigma$ is \\
    \begin{tabular}{|c|c|c|}
  \hline  & a & Not(a) \\ \hline
        1 & 2 &   \\ \hline
        2 & 3 & 1 \\ \hline
        3 & 4 &   \\ \hline
       *4 &   &   \\ \hline
    \end{tabular}
\section{Page 110 \#19}
    Rev(R) can be constructed from the DFA of R.
    Reverse the direction of all the arrows.
    Create a new state (start state). Then make a lambda transition from it to the end-states.
    Turn the old start state into an end state and the old end states into normal states.
    The resulting $\epsilon$-NFA will be Rev(R), and the existange of it proves its regularity.

\section{107 and 111 \#6 and \#23}
    6. C-style multiline comment
    /\*(\*Not(/))*\*/ \\
    23. Plurals: Does it end in an s? if the first part matches a set of words that are not regular plurals, then the ending pattern is different, with pretty unique endings. To be truly accurate, it would need to be a big set of ors. \\
    For tenses, again, match a single string such that its followed by one of the normal endings. The first group part would be the original, and the tenses from the endings.

\end{document}
