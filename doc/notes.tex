\documentclass[11pt]{article}
\usepackage{enumerate}
\begin{document}

\section{Compiler Toolchains}
    General Process
    \begin{itemize}
        \item Compiler :: Src $\rightarrow$ Object File
        \item Object Files contain machine code in a specific format (such as ELF)
        \item Linker :: Object File(s) $\rightarrow$ Machine Code
        \item Loader :: Machine Code $\rightarrow$ RAM $\rightarrow$ CPU (Running)
    \end{itemize}

    Kinds of machine code
    \begin{itemize}
        \item Pure machine code - No OS, no libraries. Just straight code being executed
        \item Augmented Machine code - Uses an OS for I/O and system libraries
        \item Virtual Machine Code - System/library abstraction layer
            Boot strapping: Small initial/temporary compiler that generates a compiler that can compile that language
            ex: C compiler for Haskell will generate basic Haskell features. Then with those features, the main (reference) compiler can be created.
    \end{itemize}
    Compiler targets
    \begin{itemize}
        \item ASM: Create Assembly Code that will later be assembled, linked, etc
            C: The universal assembler. Go to a lower level language
        \item Relocatable Binary Format: External references, data addresses
            and function addresses are unbound. They will later be bound by the
            loader. Position independent code uses relative offsets for addresses.
            This promotes modular development and cross-language support (makes
            them easier to implement, that is)
        \item Absolute Binary Format: Executable format which is directly executable.
            Mostly in embedded space. Easy to understand, because it's all RIGHT there.
    \end{itemize}

    Syntax - Structure of how it's said/expressed
    \begin{itemize}
        \item Sequence of symbols that are legal, independent of any notion of
            what they mean: CFG/CFL \\
            CFG promotes a rigid fundamental scan and tokenization process
        \item Stataic semantics: Types and type checking. Whether identifiers
            are declared (declared variables). Number of arguments for a procedure
    \end{itemize}

    Semantics - What is meant
    \begin{itemize}
        \item Can include syntactic rules that cannot be expressed by a CFG
    \end{itemize}

    Runtime Semantics
    \begin{itemize}
        \item Denotational Semantics
            Meaning of a construct in terms of its constituants: x=y    L[x]m = L[y]m
    \end{itemize}

    Compiler process:
    \begin{enumerate}
        \item Source Code
        \item $\rightarrow$ Scanner (pre-processor + tokenizer)
        \item token stream
        \item $\rightarrow$ Parser
        \item AST (Abstract Syntax Tree) - Can be decorated for type-checking
        \item $\rightarrow$ Some code options \
            Target Code Right Now \
            Optimization $\rightarrow$ Target code \
            IR (Intermediate Representation) Code. Basically, go to an abstract idea of a machine.
            Then IR $\rightarrow$ Optimizer(s) chain.
            Finally to Target-Code Generator
        \item Question: Some compiler writers use C as their target code. What some of the pros and cons of this?
    \end{enumerate}

    AC to DC:
    \begin{itemize}
        \item f b i a a = 5 b = a + 3.2 p b
        \item f b ; i a ; a = 5 ; b = a + 3.2 ; p b
        \item We have a CFG for this.
        \item Syntactically Valid: Checking static semantics
    \end{itemize}

    Recursive-Descent parsing: 1 function per CFG Production. The function implements rules based on the CFG. It may call itself or other production functions

    Implementing one of the Recursive-Descent Functions:
    \begin{verbatim}
    prodecure Expr()
        if ts.peek() = plus or ts.peek() = minus
        then
            if ts.peek() = plus
            then
                call Match(ts, plus)
            else
                if ts.peek() = minus
                then
                    call Match(ts, minus)
                end
            end
            call Val()
            call Expr()
        else
            if ts.peek() = $
            // ignore lambda production
        end
    end
    \end{verbatim}

    Abstract Syntax Trees and Parse trees
    \begin{itemize}
        \item AST has terminal nodes that contain a node-type, and terminal value. The node-types represent the functions that have been called
        \item stmts wind up being multichild-trees, with a lot of 1-height trees. Like the parse tree, the AST MUST be interpreted left-to-right
    \end{itemize}

    Scanning and Regular Expressions
    \begin{itemize}
        \item Synonamous terms: Scanner == Lexer == Lexical Analysis
        \item Scanning tries to match the absolute longest possible match
        \item Regular Expressions are a good way to do lexing
        \item Alphabet: $\Sigma$
        \item We have Token Classes, which describe the class of tokens that are matched by a particular regular expression. An instance of a token class is a particular token that matches
        \item Regex Golf (without the golfing)
        \item Definitions:
            \begin{itemize}
                \item Eol = End of Line
                \item L = All Letters
                \item D = All Digits
            \end{itemize}
        \item Example: Phone Numbers: $D^3-D^3-D^4$
        \item An integer: \
            ($\lambda$|-|+)D+
        \item A C identifier: \
            (L|\_)(D|L|\_)*
        \item Java or C++ // comments: \
            //Not(Eol)*Eol
        \item Floating Point number: \
            ($\lambda$|-|+)(D*.D+|D+.D*)
        \item Scientific Notation: \
            ($\lambda$|-|+)($\lambda$|D*.)D+(e|E)($\lambda$|-|+)D+
        \item A comment enclosed by '\#\#' on both sides: \
            \#\#Not(Eol)*\#\#
        \item An identifier that does not permit two \_'s together or a final \_:
            Not(\_)*(\_|$\lambda$)
        \item A C style multiline comment: \
            /\*(Not((?<*))Not(/)|Not(*))*\*/
        \item A simple division expression with balanced parens: \
    \end{itemize}

    Transitions in FAs can become Transducers by using an Action Table
    \begin{itemize}
        \item We have a table representing FA with transitions. With states vs input character
        \item The action table has the exact same structure, excpet that instead of moving to a state, it calls a function at the transition
    \end{itemize}

    Flex / Lex
    \begin{itemize}
        \item Structure: \
            declarations  - Data structures and types that are needed \
            \%\% \
            Optional Lex definitions. If this is empty, do not include an additional \%\% \
            ex. \
            Blank   " "
            Digits  [0-9]+
            \%\% \
            regular expression rules - Regular expressions used for tokenizaton into the types \
            ex.
            Blank+                      { return SPACE; }
            (Digits|Digits"."Digits)    { return NUM; }
            \%\% \
            subroutine definitions - functions used as outputs from regular expressions
    \end{itemize}

    RE $\to$ DFA $\to$ NFA
    Make deterministic...
    \begin{enumerate}
        \item Record start state ( {1} ) $\to$ {1,2}
        \item Find all the possible states that can be reached by all the characters, and create next set of states that can be reached
    \end{enumerate}

    Parse f(v+v)
    \begin{enumerate}
        \item E
        \item Prefix ( E )
        \item Prefix ( v Tail )
        \item Prefix ( v + E )
        \item Prefix ( v + v Tail )
        \item Prefix ( v + v $\lambda$ )
        \item f(v + v)
    \end{enumerate}
    
    for p in prods
        if lambda: true
        else if only nonterminals
            return 

    Follow:
    set of prod rules that produce the symbol ($\alpha$)
    
    Define LR Item
    - Has a production with an index into it, indicating how much has been parsed. $\lambda$ productions Have an index only to 0, meaning that there's one parse step, and only one

    Closure (LRItem)
    I is a sec of items from the grammer. It will yield another set of items
    K = dup(I)
    if $A \to \alpha \bullet B \pi$ is an item in K, AND there exists $B \to \gamma$, then add $B \to \bullet \gamma$ to K
    repeat for all in K
    return K

    Goto(I, X) where I is a set of items and X is a grammer symbol (non terminal, terminal, or terminator). It will return a new set of items
    T = $\{$ t in I | X to the right of $\bullet \}$
    T' = $\{ \textrm{t in T} |\bullet \textrm{moved to the right, after X} \}$
    return Closure(T')

    \begin{itemize}
        \item Closure ( $\{ S \to \bullet E \$ \}$ ) \\
            => $\{ S \to \bullet E \$, \\
                 E \to \bullet plus E E, \\
                 E \to \bullet num \}$ \\
            = \$1

        \item Goto (\$1, num) \\
            => $\{ \textrm{Closure} ( E \to num \bullet ) \}$ \\
            => $\{ E \to num \bullet \}$ \\
            = \$2

        \item Goto (\$1, plus) \\
            => $\{ \textrm{Closure} ( E \to plus \bullet E E) \}$ \\
            => $\{ E \to plus \bullet E E, \\
                    E \to \bullet num, \\
                    E \to \bullet plus E E \}$ \\
            = \$3

        \item Goto (\$3, E) \\
            => $\{ \textrm{Closure} ( E \to plus E \bullet E ) \}$ \\
            => $\{ E \to plus E \bullet E, \\
                 E \to \bullet num, \\
                 E \to \bullet plus E E \}$ \\
            = \$4
    \end{itemize}

    Now, bring it back in. Calculate the LR(0). Also called the LR Canonical States
    \begin{enumerate}
        \item $C = \{ Closure( \{S -> \bullet RHS, ...\} )\}$ \#Set of all the possible start states. Closure that
        \item   repeat: \\
                    m = |C| \\
                    for each I in C: \\
                        for each X (Grammer symbol): \\
                        J = Goto(I, X) \\
                        if J not empty and not in C: \\
                            add J to C \\
                until |C| == m      or until it doesnt change sizes
        \item Create an FA from this, and/or an LR(0) parse table
    \end{enumerate}

    The table: \\
    \begin{tabular}{|c|c|c|c|c|c|}
              & plus & num & \$ & S & E \\
        $I_0$ & 2    & 3   &    &   & 1 \\
        $I_1$ &      &     & acc&   &   \\
        $I_2$ & 2    & 3   &    &   & 4 \\
        $I_3$ & E -> Num                \\
        $I_4$ &      &     &    &   &   \\
        $I_5$ & E -> plus E E           \\
        $I_a$ & S -> E\$                \\
    \end{tabular}

    Read through 6.4
    p 224 \#3
    p 225 \#6 a \& c
    p 225 \#9

    p224 \#3
    Grammer:
    \begin{enumerate}[(a)]
        \item q\$ \\
            \begin{tabular}{|l|l|l|}
                0       & Init          & q\$ \\
                0       & rd $\lambda \to$ B  & Bq\$  \\
                0 B5    & sh B          & \$ \\
                0 B5 q7 & sh q          & \$ \\
                0 B5    & rd q $\to$ Q    & Q\$ \\
                0 B5 Q6 & sh Q          & \$ \\
                0       & rd BQ $\to$ A & A\$ \\
                0 A1    & sh A          & \$ \\
                0 A1    & rd $\lambda \to$ C & C\$ \\
                0 A1 C14& sh C          & \$ \\
                0       & rd AC $\to$ S & S\$ \\
                0 S4    & sh S          & \$ \\
                0 S4 \$8& sh \$         & \\
                0       & rd S\$ $\to$ Start & Start \\
                0 Start & sh Start      &
            \end{tabular}
        \item c\$ \\
            \begin{tabular}{|l|l|l|}
                0       & init          & c\$ \\
                0       & rd $\lambda \to$ B & Bc\$ \\
                0 B5    & sh B          & c\$ \\
                0 B5    & rd $\lambda \to$ Q & Qc\$ \\
                0 B5 Q6 & sh Q          & c\$ \\
                0       & rd BQ $\to$ A & Ac\$ \\
                0 A1    & sh A          & c\$ \\
                0 A1 c11& sh c          & \$ \\
                0 A1    & rd c $\to$ C  & C\$ \\
                0 A1 C14& sh C          & \$ \\
                0       & rd AC $\to$ S & S\$ \\
                0 S4    & sh S          & \$ \\
                0 S4 \$8& sh \$         & \\
                0       & rd S\$ $\to$ Start & Start \\
                0 Start & sh Start      &
            \end{tabular}
        \item adc\$ \\
            \begin{tabular}{|l|l|l|}
                0       & init          & adc\$ \\
                0 a3    & sh a          & dc\$ \\
                0 a3    & rd $\lambda \to$ B & Bdc\$ \\
                0 a3 B9 & sh B          & dc\$ \\
                0 a3 B9 & rd $\lambda \to$ C & Cdc\$ \\
                0 a3 B9 C10& sh C       & dc\$ \\
                0 a3 B9 C10 d12& sh d   & c\$ \\
                0       & rd aBCd $\to$ A & Ac\$ \\
                0 A1    & sh A          & c\$ \\
                0 A1 c11& sh c          & \$ \\
                0 A1    & rd c $\to$ C  & C\$ \\
                0 A1 C14& sh C          & \$ \\
                0       & rd AC $\to$ S & S\$ \\
                0 S4    & sh S          & \$ \\
                0 S4 \$8& sh \$         & \\
                0       & rd S\$ $\to$ Start & Start \\
                0 Start & sh Start      &
            \end{tabular}
    \end{enumerate}

    p 225 \#9
    The primary reason that this is un-ambiguous is the singular derivables in
    the right-hand side of all the rules. With each derivation, there's only
    one possible expansion choice \\

    The grammer
    \begin{itemize}
        \item $S \to E \$$
        \item $E \to E plus E$ \\
            | num
    \end{itemize}

    Creating the table
    \begin{itemize}
        \item $I0 = \{ S \to \bullet E \$, E \to \bullet E plus E, E \to \bullet num \}$
        \item $Goto(I0, E) = I1 \{ S \to E \bullet \$, E \to E \bullet plus E \}$
        \item $Goto(I1, plus) = I2 \{ E \to E plus \bullet E, E \to \bullet num \}$
        \item $Goto(I2, E) = I3 \{ E \to E plus E \bullet \}$
        \item $Goto()...$

    \end{itemize}

    Grammer which is C-like, but minimal \\

    page 139 q7 \\
    Grammer: \\
    \begin{enumerate}
        \item Start $\to$ E \$
        \item E $\to$ T plus E
        \item   | T
        \item T $\to$ T times F
        \item   | F
        \item F $\to ($ E $)$
        \item   | num
    \end{enumerate}

    \begin{enumerate}[(a)]
        \item leftmost derivation of target: num plus num times num plus num \$ \\
            Start $\to$ \\
            E \$ $\to$ \\
            T plus E \$ $\to$ \\
            F plus E \$ $\to$ \\
            num plus E \$ $\to$ \\
            num plus T plus E \$ $\to$ \\
            num plus T times F plus E \$ $\to$ \\
            num plus F times F plus E \$ $\to$ \\
            num plus num times F plus E \$ $\to$ \\
            num plus num times num plus E \$ $\to$ \\
            num plus num times num plus T \$ $\to$ \\
            num plus num times num plus F \$ $\to$ \\
            num plus num times num plus num \$
        \item Rightmost derivation of target: num times num plus num times num \$ \\
            Start $\to$ \\
            E \$ $\to$ \\
            T plus E \$ $\to$ \\
            T plus T \$ $\to$ \\
            T plus T times F \$ $\to$ \\
            T plus T times num \$ $\to$ \\
            T plus F times num \$ $\to$ \\
            T plus num times num \$ $\to$ \\
            T times F plus num times num \$ $\to$ \\
            T times num plus num times num \$ $\to$ \\
            F times num plus num times num \$ $\to$ \\
            num times num plus num times num \$ $\to$
        \item Pluses always come before times. They are the most important. Then leftmost causes Left associativity, and Right casuses Right assosiativity
    \end{enumerate}

    page 225 q6 parts b,d
    \begin{enumerate}[(a)]
        \item Skip
        \item
            \begin{enumerate}
                \item S $\to$ StmtList \$
                \item StmtList $\to$ Stmt semi StmtList
                \item   | Stmt
                \item Stmt $\to$ s
            \end{enumerate}
            This grammer is NOT $LR(0)$ because the StmtList has two rules that
            start with the same symbol. When that symbol is on the left-side, we
            don't know if we should finish the rule or if we should read the next character
        \item Skip
        \item
            \begin{enumerate}
                \item S $\to$ StmtList \$
                \item StmtList $\to$ s StTail
                \item StTail $\to$ semi StTail
                \item   | $\lambda$
            \end{enumerate}
            Yes, this grammer is LR(0). The use of tail makes it very clear that
            the first 's' should be read and rule 2 used, THEN move to the next
            character/input, from which, the rule to use can be determined based
            on only the next input.
    \end{enumerate}

    Start $\to$ S\$ $\to$ AB $\to$ aB $\to$ ab\$
    Start $\to$ S\$ $\to$ AB $\to$ aB $\to$ a\$
    Start $\to$ S\$ $\to$ ac\$
    Start $\to$ S\$ $\to$ xAc\$ $\to$ xac\$

    \subsection{LGA 4}
    Page 141 q6 \\
    Start with the set A, consisting of all symbols produced by the start symbol. \\
    For all non-terminals in the set, \\
        Add all the symbols produced by the non-terminal to the set, \\
        then remove the non-terminal from the set. \\
    repeat until no changes are made (which includes running out of non-terminals) \\
    The set of unreachable symbols $U = \Sigma - A$ \\

    Overly simple CFG \\
    PRGM -> S \\
    S -> if ( b ) S MaybeElse
    S -> s
    MaybeElse -> lambda
              | else S

    Do 2 and 3

\end{document}
